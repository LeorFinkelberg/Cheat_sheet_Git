\documentclass[%
	11pt,
	a4paper,
	utf8,
	%twocolumn
		]{article}	

\usepackage{style_packages/podvoyskiy_article_extended}


\begin{document}
\title{Наиболее полезные конструкции\\системы контроля версий  \texttt{Git}}

\author{}

\date{}
\maketitle

\thispagestyle{fancy}

\tableofcontents


\section{Предостережения}


\section{Термины и определения}

\noindent\texttt{HEAD} -- специальный \textit{указатель} на текущую \emph{локальную ветку}, которая в свою очередь ссылается на последнее зафиксированное состояние, т.е. на \emph{последний коммит}.


\section{Конструкции \texttt{Git}}

\subsection{Удаление файлов}

Удалить файл из \emph{области индексирования} и заодно удалить указанный файл из рабочей папки. Чтобы система \texttt{Git} перестала работать с файлом, его нужно удалить из числа отслеживаемых (точнее, убрать из области индексирования) и зафиксировать данное изменение 

\begin{lstlisting}[
language = cmd,
numbers = none
]
$ git rm file_name.py
\end{lstlisting}


Удалить файл из области индексирования\footnote{\texttt{Git} перестает следить за файлом, т.е. он становится \emph{неотслеживаемым}!}, но оставить его в рабочей папке. Данная команда в отличие от \texttt{git reset HEAD file\_name.py} может использоваться как до первой фиксации (\texttt{git commit}), так и после

\begin{lstlisting}[
language = cmd,
numbers = none
]
git rm --cached file_name.py
\end{lstlisting}


Удалить все файлы с расширением \texttt{.log} из директории \texttt{log}







% Источники в "Газовой промышленности" нумеруются по мере упоминания 
\begin{thebibliography}{99}\addcontentsline{toc}{section}{Список литературы}
	\bibitem{  }{  }
\end{thebibliography}

%\listoffigures\addcontentsline{toc}{section}{Список иллюстраций}

\end{document}
